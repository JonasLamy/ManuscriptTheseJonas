% !TEX root = ../my-thesis.tex
%
\chapter{Annexes}
\label{sec:appendix}

\section{Preuve de l'élongation maximale du noyau du filtre de Meijering dans le cas 3D}
\label{APP:Proof of Meijering's maximal flatness for 3D case}

\fontsize{8pt}{12pt}\selectfont

Nous étendons à la 3D, l'analyse 2D réalisée dans l'annexe de \cite{Meijering2004_neurite_vesselness}.
À translation et rotation prêt et sans perte de généralité, on suppose que la structure tubulaire est orientée dans l'axe $x$ de l'espace 3D $\mathbb R^3$. Le point d'intérêt est localisé en $\textbf{x} = \textbf{0}$.
En suivant \cite{Meijering2004_neurite_vesselness}, on cherche à obtenir  

\begin{equation}
    \lim_{\textbf{x}\to \textbf{0}} (\textbf{e}_x.\nabla)^2 h'(\textbf{x})=0
    \label{EQ:TOSOLVE}
\end{equation}
avec 
\begin{align}
h'(\textbf{x}) & = (\alpha(\textbf{e}_x.\nabla)^2 + (\textbf{e}_y.\nabla)^2 + (\textbf{e}_z.\nabla)^2)G(\textbf{x}) \nonumber \\
G(\textbf{x})& = \frac{1}{2 \pi \sigma^2}e^{-\frac{x^2 + y^2 + z^2}{2\sigma^2}} \nonumber
\end{align}
où $\{\textbf{e}_x, \textbf{e}_y, \textbf{e}_z\}$ est une base canonique de  $\mathbb  R^3$.
On a
\begin{align}
h'(\textbf{x}) & =   \alpha  \pdv{^{2}G(\textbf{x})}{x^2} + \pdv{^{2}G(\textbf{x})}{y^2} + \pdv{^{2}G(\textbf{x})}{z^2} \nonumber \\ 
& = \frac{\alpha x^2 + y^2 + z^2 - (\alpha+2) \sigma^2}{\sigma^4} G(\textbf{x}) \\
& = p(\textbf{x}).G(\textbf{x})
\nonumber
\end{align}
et
\begin{align}
    (\textbf{e}_x.\nabla)^2 h'(\textbf{x}) &= \pdv{^{2}h'(\textbf{x})}{x^2} \\
    & = \pdv{^{2}p(\textbf{x}).G(\textbf{x})}{x^2} \nonumber\\
    & = \pdv{^{2}p(\textbf{x})}{x^2} .G(\textbf{x}) + 2 \pdv{p(\textbf{x})}{x} . \pdv{G(\textbf{x})}{x} + p(\textbf{x}). \pdv{^{2}G(\textbf{x})}{x^2} \nonumber
\end{align}
Comme
\begin{align}
    &\lim_{\textbf{x}\to \textbf{0}} G(\textbf{x}) = \frac{1}{2 \pi \sigma^2}~; 
    \lim_{\textbf{x}\to \textbf{0}} \pdv{G(\textbf{x})}{x}  = 0~; 
    \lim_{\textbf{x}\to \textbf{0}} \pdv{^{2}G(\textbf{x})}{x^{2}}  = -\frac{1}{2 \pi \sigma^4} \nonumber\\
    &\lim_{\textbf{x}\to \textbf{0}} p(\textbf{x}) = - \frac{\alpha + 2}{\sigma^2}~; 
    \lim_{\textbf{x}\to \textbf{0}} \pdv{p(\textbf{x})}{x}  = 0~; 
    \lim_{\textbf{x}\to \textbf{0}} \pdv{^{2}p(\textbf{x})}{x^{2}} = \frac{2 \alpha}{\sigma^4} \nonumber
\end{align}
on en déduit
\begin{equation}
\lim_{\textbf{x}\to \textbf{0}} (\textbf{e}_x.\nabla)^2 h'(\textbf{x}) = 
\frac{3\alpha + 2}{2 \pi \sigma^6} \nonumber
\end{equation}
et Eq.~\eqref{EQ:TOSOLVE} est résolue si et seulement si $\alpha = -\frac{2}{3}$.