% !TEX root = ../my-thesis.tex
%
\pdfbookmark[0]{Abstract}{Abstract}
\addchap*{Résumé}
\label{sec:abstract}


La segmentation des vaisseaux du foie dans les images médicales reste un problème ouvert encore aujourd'hui. Celle-ci est pourtant utile pour les médecins lors de la planification opératoire. Les difficultés de la segmentation des vaisseaux hépatiques sont multiples. Les vaisseaux constituent une faible part du volume total de l'organe. Ils présentent un faible contraste par rapport aux tissus environnants, une morphologie variable et une taille approchant la limite de résolution des images pour les vaisseaux les plus fins. Le contexte d'acquisition des images hépatiques rend la tâche d'autant plus difficile, car l'imagerie tomodensitométrique (TDM) et l'imagerie par résonance magnétique (IRM) ajoutent des difficultés supplémentaires par le biais d'artefacts (bruit, hétérogénéité de contraste, mouvements, etc.). Pour faciliter la segmentation, on peut utiliser des filtres de rehaussement de vaisseaux. Ces filtres permettent d'augmenter le contraste des vaisseaux en ne filtrant que les structures tubulaires. Un certain nombre de filtres ont été proposé dans la littérature, mais en pratique on observe un recours quasi systématique aux mêmes filtres et l'utilisation de paramètres par défaut. Nous avons donc voulu comprendre la validité de ces pratiques, comparer les performances de différents filtres et étudier leur pertinence dans le contexte de l'imagerie du foie. Dans ce contexte, nous proposons un banc de test reproductible et extensible qui permet d'évaluer des filtres de rehaussement dans un ensemble de zones d'intérêt définies par un utilisateur. Nous mettons ensuite ce banc de test en pratique en analysant sept filtres de rehaussement de vaisseaux à travers trois jeux de données (TDM, IRM et IRM synthétique). Une analyse exhaustive des performances de ces filtres est proposée à travers l'étude du rehaussement dans six zones d'intérêt couvrant : l'organe, le voisinage des vaisseaux par tailles et les bifurcations des vaisseaux. Nous discutons également de l'influence des paramètres des filtres et proposons une série de recommandations d'usage en fonction du contexte. L'ensemble du code nécessaire à nos travaux est disponible publiquement : les outils d'annotations, les bases de données, l'implémentation des filtres ainsi qu'une démonstration en ligne. Cette thèse regroupe donc un ensemble de connaissances et d'outils pour toute personne souhaitant travailler avec les filtres de rehaussement de vaisseaux.

\pagebreak

{\usekomafont{chapter}Abstract}
\label{sec:abstract-diff}

Hepatic vessels segmentation in medical images is still an open problem. Yet, such segmentation is useful for praticians in a couple of contexts including surgical planning. The difficulties of hepatic vessels segmentation are numerous. Compared to the volume of the liver, the vessels cover a  sparser area. They exhibit a weaker contrast compared to neighbouring tissues, a variable morphology and for the smallest vessels, a size nearing the images resolution. The imaging context of computed tomography (CT) and magnetic resonance imaging (MRI) adds an extra layer of difficulties induced by specific artifacts (noise, contrast inhomogeneity, movement, etc.).
To ease the segmentation process, one can use vesselness filters. These filters can improve the contrast of vessels by enhancing tubular structures. A number of these filters has been proposed in the litterature, however the usage of the same filters with default parameters has been observed in the majority of cases. In this context, our objectives have been to understand this trend, to compare the performances of several filters and to study their effectiveness in the context of liver vessels imaging. We chose to build a reproducible and extensible benchmark able to evaluate the filters over areas of interest defined by the user. We then put this benchmark in practice by analysing seven vesselness filters over tree datasets (CT scans, MRI, and synthetic MRI). An exhaustive analysis of the filters is performed over six areas of interest: the organ, the vessels neighbourhood by size and the bifurcations. We also discuss the effect of the parameters of the filters and give usage recommendations depending on the segmentation context. All the code needed for our experiments is publicly available: the annotations tools, datasets and ground truth as well as filters implementation and an online demonstrator. This PhD thus regroups a comprehensive set of knowledge and tools for anyone looking to use vesselness filters in their work. 

