% !TEX root = ../my-thesis.tex
%
\pdfbookmark[0]{Remerciements}{Remerciements}
\addchap*{Remerciements}
\label{sec:Remerciements}


% Marie cath
% Juliette polo
% Les lyonnais + Sara, laura
% Labo, creatis, 
% Les directeurs de thèse
% LIRIS pour financement

Je voudrais remercier en premier ma maman, Marie-Claire, ainsi que ma tante, Catherine, qui me soutiennent depuis toujours. Merci de croire en moi et de m'avoir toujours aidé à réaliser mes ambitions. 

Merci à Juliette et Paul-Antoine pour m'avoir soutenu et aidé à me lancer dans cette aventure. Cette thèse a été pour moi une expérience très riche tout au long de ces quatre années et je n'aurais peut-être pas fait le pas sans vos encouragements.

Merci aux copains Lyonnais et anciens Lyonnais : Flo, Mathou, Plach, Vincent, Quentin, Romain, Sassou, Chloé ainsi que Laura et Sara pour toutes les soirées, sorties, raclettes, pique-niques, crémaillères, anniversaires, déménagements, etc. Merci surtout d'avoir fait de Lyon une seconde maison où je me suis toujours senti chez moi ! Merci aussi aux Slack et aux copains du JDR pour les nombreuses soirées jeux qui m'ont permis de décompresser, en particulier pendant le confinement. 

Je voudrais aussi remercier l'ensemble des permanents et doctorants du laboratoire du LIRIS de Lyon~2. Merci pour les groupes de travail, les sorties d'équipe, les discussions informelles et tous les petits moments qui participent à créer cette ambiance chaleureuse et familiale du LIRIS à Lyon~2. Un merci spécial aux doctorants du bureau 114 et en particulier Rémi, Clément, Dev et Romain pour les brainstormings autour du tableau, les petits coups de pouce et les moments beaucoup moins sérieux.

Une petite pensée pour les "7 merveilles d'Odyssée" de CREATIS pour les discussions sur la segmentation des vaisseaux, pour avoir supporté mes blagues et m'avoir écouté plus d'une fois râler. Dédicace spéciale à Sophie pour avoir vérifié mes résultats et pour tout le reste ;)

Merci aussi aux membres du projet ANR R-Vessel-X avec qui j'ai pu travailler et échanger tout au long de ces quatre dernières années. Nos discussions ont toujours été très enrichissantes sur le plan humain et scientifique.

Enfin, je voudrais dire un grand merci à mes encadrants de thèse Bertrand, Nicolas et Odyssée. Merci d'avoir toujours été présents tout au long de ces quatre années et d'avoir chacun apporté un regard différent, mais toujours pertinent, sur mes travaux. Merci aussi pour avoir toujours été disponibles et à l'écoute. J'ai à chaque fois pu bénéficier d'un regard critique et bienveillant sur mes idées, même lorsque celles-ci étaient peu claires ou farfelues. J'espère que mon travail ainsi que ce manuscrit reflètent tout ce que j'ai pu assimiler et réaliser grâce à vous.

Merci aussi à la Direction du LIRIS pour m'avoir accordé un financement supplémentaire de fin de thèse.

%Cette thèse a été financé par l'Agence Nationale de Recherche via le projet R-Vessel-X : ANR-18-CE45-0018, ANR-18-CE45-0014, ANR-20-
%CE45-0011.

%Deux mois ont été financé grâce au fond d'aides aux doctorants en fin de thèse du laboratoire LIRIS.


