% !TEX root = my-thesis.tex


% **************************************************
% Files' Character Encoding
% **************************************************
\PassOptionsToPackage{utf8}{inputenc}
\usepackage{inputenc}



% **************************************************
% Information and Commands for Reuse
% **************************************************
\newcommand{\thesisTitle}{Rehaussement de vaisseaux sanguins appliqué aux images médicales 3D}
\newcommand{\thesisName}{Jonas Lamy}
\newcommand{\thesisSubject}{Informatique et traitement d'images}
\newcommand{\thesisDate}{\today}
\newcommand{\thesisVersion}{V3}

\newcommand{\thesisFirstReviewer}{XXX}
\newcommand{\thesisFirstReviewerUniversity}{\protect{XXX University}}
\newcommand{\thesisFirstReviewerDepartment}{Department of XXX}

\newcommand{\thesisSecondReviewer}{XXX}
\newcommand{\thesisSecondReviewerUniversity}{\protect{XXX University}}
\newcommand{\thesisSecondReviewerDepartment}{Department of XXX}

\newcommand{\thesisThirdReviewer}{XXX}

\newcommand{\thesisFirstSupervisor}{Bertrand KERAUTRET}
\newcommand{\thesisSecondSupervisor}{Nicolas PASSAT}
\newcommand{\thesisThirdSupervisor}{Odyssée MERVEILLE}

\newcommand{\thesisUniversity}{\protect{Université Lyon 2}}
\newcommand{\thesisUniversityDepartment}{LIRIS}
\newcommand{\thesisUniversityInstitute}{LIRIS}
\newcommand{\thesisUniversityGroup}{Clean Thesis Group (CTG)}
\newcommand{\thesisUniversityCity}{Lyon}
\newcommand{\thesisUniversityStreetAddress}{LIRIS, Bât C, 5 avenue Pierre Mendes France, Bron }
\newcommand{\thesisUniversityPostalCode}{69000}


\newcommand{\chapIntroN}{\ref{sec:introduction}}
\newcommand{\chapContextN}{\ref{sec:contexte}}
\newcommand{\chapSOTAN}{\ref{sec:SOTA}}
\newcommand{\chapBenchN}{\ref{sec:Benchmark}}
\newcommand{\chapAnalysisN}{\ref{sec:Analysis}}
\newcommand{\chapReproN}{\ref{sec:reproductibility}}
\newcommand{\chapEndN}{\ref{sec:Ending}}
\newcommand{\chapTotalN}{7}

\newcommand{\chapIntro}{un}
\newcommand{\chapContext}{deux}
\newcommand{\chapSOTA}{trois}
\newcommand{\chapBench}{quatre}
\newcommand{\chapAnalysis}{cinq}
\newcommand{\chapRepro}{six}
\newcommand{\chapEnd}{sept}
\newcommand{\chapTotal}{sept}

\newcommand{\plugIns}{plugins}
\newcommand{\plugIn}{plugin}

% **************************************************
% Debug LaTeX Information
% **************************************************
%\listfiles


% **************************************************
% Load and Configure Packages
% **************************************************
\usepackage[english,french]{babel} % babel system, adjust the language of the content
\PassOptionsToPackage{% setup clean thesis style
    figuresep=colon,%
    hangfigurecaption=false,%
    hangsection=true,%
    hangsubsection=true,%
    sansserif=false,%
    configurelistings=true,%
    colorize=full,%
    colortheme=bluemagenta,%
    configurebiblatex=true,%
    bibsys=biber,%
    bibfile=TheseJonas,%
    bibstyle=alphabetic,%
    bibsorting=nty,%
}{cleanthesis}
\usepackage{cleanthesis}

\hypersetup{% setup the hyperref-package options
    pdftitle={\thesisTitle},    %   -   title (PDF meta)
    pdfsubject={\thesisSubject},%   - subject (PDF meta)
    pdfauthor={\thesisName},    %   - author (PDF meta)
    plainpages=false,           %   -
    colorlinks=false,           %   - colorize links?
    pdfborder={0 0 0},          %   -
    breaklinks=true,            %   - allow line break inside links
    bookmarksnumbered=true,     %
    bookmarksopen=true          %
}

% **************************************************
% Other Packages
% **************************************************
%\usepackage[french]{babel}
\usepackage{scrhack}
\usepackage{amsmath}
\usepackage{amssymb}
\usepackage{amsfonts}
\usepackage{tikz}
%\usepackage{subfigure}
\usepackage{subcaption}
\usepackage{graphics}
\usepackage{url}
\usepackage{algorithm}
\usepackage{algpseudocode}
\usepackage{hyperref}
\usepackage{multirow}
\usepackage{adjustbox}
\usepackage{comment}
%\usepackage[active,tightpage,floats]{preview}
\usetikzlibrary{shapes.geometric, arrows}

% ********************************
% Jonas command
% ********************************
\newcommand{\todo}[1]{\textcolor{red}{\textbf{TODO:}#1}}
\newcommand{\todoNote}[1]{\textcolor{red}{#1}}
\newcommand{\newV}[1]{\textcolor{black}{#1}}
\newcommand{\oldV}[1]{\textcolor{lightgray}{#1}}
\newcommand{\etal}{ \emph{et~al.} \xspace}
\newcommand{\tbf}[1]{\textbf{#1}}

\newcommand{\percent}{\% }

% partial
\newcommand{\pdv}[2]{\frac{\partial #1}{\partial #2}}

% Masks bifurcation
\newcommand{\maskbif}{{\ensuremath{M_{\textrm{bif}}}}\xspace}
% Masks global
\newcommand{\maskglobal}{{\ensuremath{M_{\textrm{glo}}}}\xspace}
% Masks vascular
\newcommand{\maskvascular}{{\ensuremath{M_{\textrm{vasc}}}}\xspace}
% Masks vessell
\newcommand{\maskvessel}{{\ensuremath{M_{\textrm{vess}}}}\xspace}
% Masks vessell medium
\newcommand{\maskvesselMedium}{{\ensuremath{M_{\textrm{vess}}^{\textrm{medium}}}}\xspace}
% Masks vessell large
\newcommand{\maskvesselLarge}{{\ensuremath{M_{\textrm{vess}}^{\textrm{large}}}}\xspace}
% Masks vessell small
\newcommand{\maskvesselSmall}{{\ensuremath{M_{\textrm{vess}}^{\textrm{small}}}}\xspace}
% French algorithm

\renewcommand{\listalgorithmname}{Liste des algorithmes}
\floatname{algorithm}{Algorithme}
\renewcommand{\algorithmicreturn}{\textbf{retourne}}
\renewcommand{\algorithmicprocedure}{\textbf{procédure}}
%\renewcommand{\Not}{\textbf{non}\ }
%\renewcommand{\And}{\textbf{et}\ }
%\renewcommand{\Or}{\textbf{ou}\ }
\renewcommand{\algorithmicrequire}{\textbf{Entrée:}}
\renewcommand{\algorithmicensure}{\textbf{Sortie:}}
%\renewcommand{\algorithmiccomment}[1]{\{#1\}}
\renewcommand{\algorithmicend}{\textbf{fin}}
\renewcommand{\algorithmicif}{\textbf{si}}
\renewcommand{\algorithmicthen}{\textbf{alors}}
\renewcommand{\algorithmicelse}{\textbf{sinon}}
\renewcommand{\algorithmicfor}{\textbf{pour}}
\renewcommand{\algorithmicforall}{\textbf{pour tout}}
\renewcommand{\algorithmicdo}{\textbf{faire}}
\renewcommand{\algorithmicwhile}{\textbf{tant que}}
\newcommand{\algorithmicelsif}{\algorithmicelse\ \algorithmicif}
\newcommand{\algorithmicendif}{\algorithmicend\ \algorithmicif}
\newcommand{\algorithmicendfor}{\algorithmicend\ \algorithmicfor}

%\usepackage[showframe]{geometry}

% adding subsubsubsection for Chap3.
\usepackage{titlesec}

\titleclass{\subsubsubsection}{straight}[\subsection]
\newcounter{subsubsubsection}[subsubsection]
\renewcommand\thesubsubsubsection{\thesubsubsection.\arabic{subsubsubsection}}
\renewcommand\theparagraph{\thesubsubsubsection.\arabic{paragraph}} % optional; useful if paragraphs are to be numbered

\titleformat{\subsubsubsection}
  {\normalfont\normalsize\bfseries}{\thesubsubsubsection}{1em}{}
\titlespacing*{\subsubsubsection}
{0pt}{3.25ex plus 1ex minus .2ex}{1.5ex plus .2ex}


\makeatletter
\renewcommand\paragraph{\@startsection{paragraph}{5}{\z@}%
  {3.25ex \@plus1ex \@minus.2ex}%
  {-1em}%
  {\normalfont\normalsize\bfseries}}
\renewcommand\subparagraph{\@startsection{subparagraph}{6}{\parindent}%
  {3.25ex \@plus1ex \@minus .2ex}%
  {-1em}%
  {\normalfont\normalsize\bfseries}}
\def\toclevel@subsubsubsection{4}
\def\toclevel@paragraph{5}
\def\toclevel@paragraph{6}
\def\l@subsubsubsection{\@dottedtocline{4}{7em}{4em}}
\def\l@paragraph{\@dottedtocline{5}{10em}{5em}}
\def\l@subparagraph{\@dottedtocline{6}{14em}{6em}}
\makeatother

\setcounter{secnumdepth}{4}
\setcounter{tocdepth}{4}

%\makeatletter
%\setcounter{secnumdepth}{4}
%\titlespacing{\subsubsubsection}{0em}{.5em}{0em}%[0pt]
%\ifct@cthesis@hangsubsection
%        \titleformat{\subsubsubsection}[hang]%
%            {\usekomafont{subsubsection}}%
%            {\color{ctcolorblack}\thesubsubsubsection\hspace*{10pt}}%
%            {0pt}%
%            {\raggedright}%
%            [\phantomsection]
%\else
%        \titleformat{\subsubsubsection}[block]%
%            {\usekomafont{subsubsection}}%
%            {\color{ctcolorblack}\thesubsubsubsection\hspace*{10pt}}%
%            {0pt}%
%            {\raggedright}%
%            [\phantomsection]
%\fi
%\makeatother
